Working towards a PhD is an incredibly humbling and transformative journey. There are many people who have made this journey possible and rewarding for me. I'm extremely grateful to my adviser, Amit Singer, for being a great mentor and for his patience, wisdom, and enthusiasm. When I made the somewhat late transition from theoretical physics to applied mathematics in the third year of graduate school, Amit welcomed me to the many interesting challenges in the world of cryo-EM. I've learnt a great deal from our technical discussions and the work in this thesis would have been impossible without Amit's valuable insights and suggestions. I thank him for allowing me enough freedom and time to solve problems, while ensuring that I did not go astray. I thank my physics advisor, Joshua Shaevitz, for his interest and enthusiasm about my work.
I'd like to acknowledge my committee members and readers for their valuable inputs and comments.
I'm deeply indebted to Paul Steinhardt and Salvatore Torquato. My journey in graduate school began with dabbling in condensed matter physics problems under their guidance. I cherish all the interesting problems concerning hyperuniformity and quasicrystals that I learnt from them. I'm grateful to Paul for encouraging me to recognize and pursue a different research direction in cryo-EM.

I had a wonderful experience working in William Happer's atomic physics lab as part of my experimental qualifying exams. I thank my colleagues in Will's group, Iannis, Bart, and Robert, for helping the theorist me figure out how to work on experiments in the laboratory.
I would like to thank my collaborators, Teng Zhang and Jane Zhao, from whom I have learnt a great deal. Many thanks to Fred Sigworth and his group for organizing a tour of the electron microscopes at Yale, and his many helpful suggestions about my work. I'm grateful to Adam Frost, and his students, for giving me an inside look into the electron microscopes at UCSF. I'm very thankful for collaborations and enlightening discussions with Yoel Shkolnisky, Joakim Anden, Xiuyuan Cheng, and Lanhui Wang. I'm grateful to Herman Verlinde, the physics department chair, for helpful discussions about navigating through grad school.

I would like to express my deepest gratitude to all my teachers, friends, and colleagues throughout school and undergraduate studies - it hard to do justice in this limited space to the role that all of them have played in shaping my thoughts. My first encounter with research in theoretical physics was through a summer project with Urjit Yajnik, which resumed later for my senior thesis. I thoroughly enjoyed our conversations on cosmology and physics in general. It is hard to exaggerate the role that Raghava Varma and Achim Kempf have played in my decision to pursue a PhD. Raghava Varma motivated me tremendously to pursue research in his role as my junior thesis advisor, and made a prophetic comment in passing that I would enjoy research in applied math better, which I came to realize for myself only much later in grad school. I worked as a research assistant in Kempf's lab at the University of Waterloo, Canada. This led to a very fruitful and enjoyable collaboration on a quantum gravity problem, and many frequent trips to Waterloo followed. I have had the pleasure of many fascinating conversations with him about physics, history, Bollywood movies, and traveling the world! My sincere thanks to him and Dushyantha for being such fun and gracious hosts, and sightseeing Montreal with me.

Graduate school is not just about research. I'm indebted to Shefalika Gandhi, whose mindfulness and meditation workshops in Princeton have been a steady source of comfort and relief. I feel lucky to have been surrounded by many wonderful friends who have made my time at Princeton memorable - Chinmay Khandekar, Debajit Bhattacharya, Jahnavi Punekar, Jaya Khanna, Nayana Prasad, Paula Mateo, Pathikrit Bhattacharya, Ravi Tandon, Sneha Rath, Srinivas Narayana (NG), Tanya Gupta, Yogesh Goyal. I feel blessed to have had lovely roommates who are `almost-sisters’ to me now - thank you Jahnavi Punekar and Nayana Prasad, for being there every single day. I cherish all the fun we had, the food we cooked and devoured tirelessly, and also all the hours we spent complaining about and over-analyzing grad school. Special thanks to Srinivas for giving me company when we worked on our respective deadlines, and for tolerating my never ending philosophical questions. I've enjoyed many musical evenings singing with Srinivas, Debajit, Jahnavi, with our other friends as the audience (sometimes forcefully). Jaya and Pathikrit have been a great source of comfort and fun - thank you for all the dinners, lunches, and music. I miss all the delightful girls' nights with Jahnavi and Paula, and Nayana, Sneha and Tanya. Yogesh and Debajit patiently trained me to shoot three pointers on the basketball court. Thanks to Yogesh for all the fights and annoying me to no end but always making up for it with a cup of the most perfect chai. Discussions with Chinmay, NG, Ravi about philosophy, life, and everything else were (and are) always intense and stimulating. Skype sessions and chats with Anasuya Mandal brightened my days and always made me smile. Thanks to Yash Deshpande for sharing his office space with me at Stanford, and for all the uplifting dog pictures. I spent a summer in Seattle which Arnab and Shaoni Sinha made memorable by their company and hospitality. 

I have enjoyed the company of everyone in Amit's group - lunches with my groupmates Amit, Joao, Jose, Susannah, Yuehaw, Yuan, Yutong, provided the much needed break from research in the office. I'm thankful to all the wonderful staff of PACM and the physics department who have been of great help - Angela Lewis, Audrey Mainzer, Ben Rose, Charlene Borsack, Darryl Johnson, Jessica Heslin, Kate Brosowsky, Laura Deevey, Lisa Giblin, Vinod Gupta.

A special thanks to my favorite coffee shops (and the physics and math coffee lounge) in Princeton and Mountain View that have powered the work in this thesis.

None of this would have been possible without the support of my family. My late grandmother, Snehalata Padalkar, was my constant companion and biggest champion growing up. This thesis is dedicated to her boundless faith and love. My parents, Santosh and Bharati Bhamre, have relentlessly supported and encouraged me to pursue my dreams. I cannot possibly convey my gratitude towards them in words. I thank my in-laws for their love. My biggest pillar of strength throughout PhD has been my best friend and now husband, Saket, who has been there with me through it all, despite the 3000 miles between us. It's impossible to put into words what he means to me, but thankfully I think he already knows. Thank you for braving through the deadlines, stressful months and the accompanying crankiness, walking with me every step of the way, taking over the chores, cheering me up, and most importantly, never letting me lose sight of the bigger, wonderful landscape of life, with you. 
